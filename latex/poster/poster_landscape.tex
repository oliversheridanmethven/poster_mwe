\RequirePackage[l2tabu, orthodox]{nag}
\documentclass[a0paper,landscape,fontscale=0.35]{baposter}

\usepackage{amsfonts,amsmath,amssymb}
\usepackage{graphicx,subcaption}

\usepackage{relsize}
\usepackage{natbib}

\usepackage{microtype}

%%% Define a caption font
\newcommand{\mycaption}[1]{
  {
    \smaller
    \emph{#1}
  }
}

%%% Color Definitions %%%%%%%%%%%%%%%%%%%%%%%%%%%%%%%%%%%%%%%%%%%%%%%%%%%%%%%%%
\definecolor{oxford_blue}{RGB}{14,31,71}
\definecolor{oxford_border}{RGB}{14,31,71}

\begin{document}
\typeout{Poster rendering started}

\begin{poster}
{
    % Show grid to help with alignment
    grid=false,
    % Column spacing
    colspacing=0.7em,
    % Color style
    %headerColorOne=cyan!20!white!90!black,
    %borderColor=cyan!30!white!90!black,
    headerColorOne=oxford_blue,
    borderColor=oxford_border,
    headerFontColor=white,
    % Format of textbox
    textborder=faded,
    % Format of text header
    headerborder=open,
    headershape=roundedright,
    headershade=plain,
    background=none,
    bgColorOne=cyan!10!white,
    headerheight=0.12\textheight
}
% Eye Catcher: Oxford logo and personal picture
{
  \makebox[0.23\textwidth]{
    \begin{tabular}{cc}
    \includegraphics[height=0.10\textheight]{./img/oxlogo}
      &
    \includegraphics[height=0.10\textheight]{./img/profile}
    \end{tabular}
    \hfill
  }
}
%%% Title %%%%%%%%%%%%%%%%%%%%%%%%%%%%%%%%%%%%%%%%%%%%%%%%%%%%%%%%%%%%%%%%%%%%%
{\textsc{Price-management under uncertainty}
}
%%% Authors %%%%%%%%%%%%%%%%%%%%%%%%%%%%%%%%%%%%%%%%%%%%%%%%%%%%%%%%%%%%%%%%%%%
{
  \vspace{0.5em}
  Asbj{\o}rn Nilsen Riseth --- \texttt{riseth@maths.ox.ac.uk}\\[0.1em]
  { Mathematical Institute, University of Oxford}
}
%%% InFoMM Logo %%%%%%%%%%%%%%%%%%%%%%%%%%%%%%%%%%%%%%%%%%%%%%%%%%%%%%%%%%%%%%%
{
  \makebox[0.23\textwidth]{
    \includegraphics[height=0.10\textheight]{./img/InFoMM}
  }
}

\headerbox{How do you decide the price of your products?}{name=introduction,span = 2,column=0,row=0}{
  \begin{minipage}[h]{.7\textwidth}
    Retail managers must decide the prices of hundreds of products
    based on company policy, revenue and profit maximisation. We will
    \begin{itemize}
    \item model the demand dependence on price changes to aide in
      the decision making process,
    \item argue for a multiobjective or ``soft'' view of constraints,
    \item define a concise framework for decision making under uncertainty.
    \end{itemize}

    Given a product unit price, $P$, and product demand per unit time,
    $V(P)$, the \emph{elasticity} $\beta$ represents how demand
    changes with price \citep{ryzin2012}. If we assume constant
    elasticity, then
    \begin{align}
      V(P) = \alpha P^\beta,
      &&\text{where } \beta = \frac{\mathrm d V/V}{\mathrm d P/P}.
    \end{align}
    The scale factor $\alpha$ can be used to incorporate
    time-dependent factors like seasonality. The price of one product
    can affect demand for other products, which is modelled similarly
    with \emph{cross-elasticities.}
    Revenue and profit are central to businesses, and are defined per
    item as
    \begin{align}
      \mbox{Revenue}(P) &= PV(P),\\
      \mbox{Profit}(P) &= (P-C)V(P),& \text{given \emph{marginal cost} C}
    \end{align}
  \end{minipage}%
  \begin{minipage}[h]{.3\textwidth}
    \centering
    \includegraphics[width=0.85\textwidth]{./img/single_revenue}
    \\[0.5em]
    \includegraphics[width=0.85\textwidth]{./img/multiple_profit}\\
    \mycaption{Total profit for two products}
  \end{minipage}%
}

\headerbox{Multiobjective optimisation}
{name=multiobjective, column=0, below=introduction, above=bottom, span=2}
{
  \begin{minipage}[h]{0.7\textwidth}
    Typical business decisions aim to optimise an objective $f_1(x)$,
    given a target $f_2(x)$. Maximising revenue without sacrificing too
    much profit can be expressed in a general optimisation form. Given
    prices $S\subset \mathbb R^n$, define
    $f_1(x) = -\sum_i \mbox{Revenue}_i(x)$, $f_2(x) =
    \nobreak Pr_{\min}-\nobreak \sum_i\mbox{Profit}_i(x)$, and solve
    \begin{align}
      \min_{x\in S}\{f_1(x)\mid f_2(x)\leq 0\}.
    \end{align}

    Constraints should be viewed as objectives, to emphasise that
    tradeoffs are being made.
    Given objectives $f_1,f_2$,
    a \emph{Pareto optimal} point $x^*\in S$ is one such
    that there are no feasible $x$ that can improve one objective
    without sacrificing the other. The set of all such points is
    called the \emph{Pareto front}. Active constraints force your
    position on the front.\\
    % With an active constraint
    % $Pr_{min}$, the decision maker has implicitly created a bi-objective
    % optimisation problem and chosen a position on the Pareto front.

    \begin{minipage}{\textwidth}
      \begin{minipage}{0.5\textwidth}
        \centering
        \includegraphics[width=0.7\textwidth]{./img/pareto_prof_rev_ws}\\
        \mycaption{Weighted Sum, 15 points. Overlaps}
      \end{minipage}%
      \begin{minipage}{0.5\textwidth}
        \centering
        \includegraphics[width=0.7\textwidth]{./img/pareto_prof_rev_nbi}\\
        \mycaption{NBI method, 15 points. Uniform spacing}
      \end{minipage}
    \end{minipage}
  \end{minipage}%
  \begin{minipage}[h]{0.3\textwidth}
    \centering
    \includegraphics[width=0.85\textwidth]{./img/pareto_drawing}
    \\[0.5em]
    \includegraphics[width=0.85\textwidth]{./img/nbi_drawing2}\\
    \mycaption{NBI subproblem}
  \end{minipage}\\[1em]

  Single-objective reformulations of the problem can be used to
  trace out the Pareto front.
  \begin{itemize}
  \item \emph{Weighted Sum} associates a relative importance weighting
    $\lambda$ with one
    objective. It has issues including a nonlinear decision
    response in $\lambda$,
    the same decision is made with different weights, and
    it cannot find concave regions of the front.
  \item The \emph{Normal Boundary Intersection} method
    \citep{das1998normal}
    generates a uniformly spaced set of Pareto points by moving
    in a direction $\hat n$ away from the \textcolor{red}{convex hull of individual minima},
    generated by the matrix $\Phi$ with columns $F(x_i^*)-F^*$.
  \end{itemize}
  \vspace{-0.8em}
  \begin{align}
    &
      \min_{x\in S}\{\lambda f_1(x)+(1-\lambda)f_2(x)\}
    &
      \max_{x\in S, t\in \mathbb R} \{t\mid \Phi\beta+t\hat n =
      F(x)-F^*\}
    &
  \end{align}
}

\headerbox{Decisions that depend on future events}
{name=uncertainty, column=2, row=0, span=2}
{
  \begin{minipage}[h]{0.7\textwidth}
    With incomplete information about our system, we model the
    uncertainty with random variables. This raises questions like:
    \begin{itemize}
    \item How do we make a preference ordering between choices? The
      objective has several outcomes for each decision.
    \item What does feasibility mean? Violations of constraints
      are only known after we make our decision.
    \end{itemize}
    A motivating example in retail is the uncertainty of marginal cost
    for products. Calculations dependent on cost can appear in
    both the objective and the constraints.
    We make the problem deterministic with \emph{risk measures},
    functionals that represents the decision maker's attitude to risk.
    With the viewpoint of constraint violations as tradeoffs, this
    allows us to define feasibility in terms of the risk measure.
  \end{minipage}%
  \begin{minipage}[h]{0.3\textwidth}
    \centering
    \includegraphics[width=0.85\textwidth]{./img/single_profit_multiple_costs}\\
    \mycaption{Profit depends on uncertain cost}
  \end{minipage}\\[1em]

  \begin{minipage}[h]{\textwidth}
    We can for example balance optimising
    expected value and minimising the standard deviation.
    This ensures a decision is sufficiently stable, and that we are
    within a safe distance from violating constraints. The balance
    between expected value and stability fits naturally in the
    multiobjective framework. The following plots show results from
    maximising profit of two products:
    \\[0.7em]
    \begin{minipage}{\textwidth}
      \begin{minipage}{0.5\textwidth}
        \centering
        \includegraphics[height=0.15\textheight]{./img/pareto_std_prof_2}\\
        \mycaption{Pareto front with Weighted Sum}
      \end{minipage}%
      \begin{minipage}{0.5\textwidth}
        \centering
        \includegraphics[height=0.15\textheight]{./img/profit_distribution_2}\\
        \mycaption{Distributions of profit, assuming costs are Gaussian}
      \end{minipage}
    \end{minipage}
  \end{minipage}\\[1em]

  \begin{minipage}[h]{0.7\textwidth}
    Standard deviation penalises better-than-average outcomes, and
    does not take into account how much we violate constraints.
    \emph{Conditional Value at Risk} ($\mbox{CVaR}_\alpha$) solves these issues by
    calculating the expected loss of the worst $1-\alpha$ proportion
    of outcomes \citep{rockafellar2007coherent}.
    \begin{align}
      \mbox{CVaR}_\alpha(X) &= \mathbb E[X\mid X\geq F_X^{-1}(\alpha)],
      & F_X \text{~is the c.d.f.~of X.}
    \end{align}
    CVaR is continuous in $\alpha$ and cover risk attitudes from
    neutral to worst-case scenarios.
    \begin{align}
      \lim_{\alpha\to0}\mbox{CVaR}_\alpha(X)&=\mathbb E[X]
      & \text{risk-neutral}\\
      \lim_{\alpha\to1}\mbox{CVaR}_\alpha(X) &= \sup X
      & \text{worst-case}
    \end{align}
  \end{minipage}%
  \begin{minipage}[h]{0.3\textwidth}
    \centering
    \includegraphics[width=0.90\textwidth]{./img/cvar_tail_drawing}\\
    \mycaption{Conditional Value at Risk takes tail risk into account}
  \end{minipage}
}

\headerbox{Further work}
{name=furtherwork, column=2, below=uncertainty, span=2}
{
  We want to model marginal costs as stochastic processes, and run
  multi-week price optimisations based on real product data.
  This will let us investigate how much decisions change with different
  risk measures.
}

\headerbox{Acknowledgements}
{name=acknowledgements, column=3, below=furtherwork, above=bottom, span=1}
{
    \vspace{1em}
    \begin{center}
      \begin{tabular}{cc}
        \includegraphics[width=0.4\textwidth]{./img/epsrc_logo.eps}
        &
          \includegraphics[width=0.4\textwidth]{./img/company_logo.eps}
      \end{tabular}
    \end{center}
}

\headerbox{References}
{name=references, column=2, below=furtherwork, above=bottom, span=1}
{
  \smaller                                  % Make the whole text smaller
  \bibliographystyle{abbrv}                 % Use plain style
  \renewcommand{\section}[2]{\vspace{0.05em}}	% Omit "References" title
  \bibliography{references}
}

\end{poster}
\end{document}

%%% Local Variables:
%%% mode: latex
%%% TeX-master: t
%%% End:
